\begin{comment}
    Figure~\ref{fig:all-methods-comparison} presents a comparison of SPO+, Rank Loss, and the proposed sigmoid surrogate on nonlinear datasets with uneven task durations.
The x-axis shows the noise standard deviation, and the y-axis plots the logarithmic SCT error relative to the optimal solution.
The sigmoid surrogate demonstrates robustness across noise levels, consistently achieving lower SCT compared to SPO+ and Rank Loss.

Figure~\ref{fig:improvement-percentages} shows the percentage improvement of SPO+ and Rank Loss relative to the sigmoid surrogate.
Although SPO+ and Rank Loss perform competitively under low-noise conditions, their relative performance deteriorates as noise increases, underscoring the robustness of the proposed approach.

Figure~\ref{fig:range-impact} examines performance as task duration ranges vary under a fixed noise level.
The sigmoid surrogate maintains lower SCT errors across wide variability, highlighting its effectiveness in challenging scheduling scenarios.
Our results demonstrate that the proposed sigmoid-based surrogate consistently outperforms SPO+ and Rank Loss in minimizing SCT, particularly under high-noise and high-variability conditions. 

Turning to efficiency analysis, Table~\ref{tab:runtimes} reports the average runtime over 10 iterations for each method under consistent settings.
The sigmoid method averaged 58.3 seconds, SPO+ averaged 57.3 seconds, and Rank Loss averaged 58.2 seconds, demonstrating comparable computational costs across methods.
While runtime is comparable across methods, the improved SCT minimization offered by the sigmoid surrogate positions it as a promising choice for real-world scheduling applications involving uncertainty and variability.


\begin{figure}[t]
    \centering
    \includegraphics[width=0.45\textwidth]{improvement--percentage.png}
    \caption{Percentage improvement of SPO+ and Rank Loss relative to the sigmoid surrogate.
    Positive values indicate worse performance compared to the sigmoid surrogate loss.}
    \label{fig:improvement-percentages}
\end{figure}

\begin{figure}[t]
    \centering
    \includegraphics[width=0.45\textwidth]{fixed--noise (2).png}
    \caption{Impact of task duration range variability on SCT error for the three methods.}
    \label{fig:range-impact}
\end{figure}

\begin{table}[t]
\centering
\caption{Average runtime over 10 iterations per method.}
\label{tab:runtimes}
\begin{tabular}{lcc}
\toprule
Method & Runtime (seconds) \\
\midrule
Sigmoid surrogate & 58.3 \\
SPO+ surrogate & 57.3 \\
Rank Loss & 58.2 \\
\bottomrule
\end{tabular}
\end{table}

\end{comment}



%\section{Experiments: Two-Machine Flow Shop}
%\label{sec:flowshop-experiments}

%Designing and evaluating surrogates for the two-machine flow shop (Problem~2) follows the same principles as in the SCT case but requires careful handling of both group assignments and within-group orderings.
%In this version of the paper, we focus on presenting the surrogate construction and leave a full empirical study of the flow shop surrogate to future work. Below we outline the intended experimental design to clarify how our framework extends empirically to this setting.